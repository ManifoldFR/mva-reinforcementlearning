\documentclass[../course-notes.tex]{subfiles}

\begin{document}

\chapter{Markov Decision Processes}


\begin{defn}[Markov Decision process]
	A \emph{Markov decision process} (MDP) is given by its state space $\calS$, action space $\calA$, state dynamics, reward structure and discount factor. In a MDP, the state dynamics can be written as
	\[
		\PP(s_{t+1} \mid s_t, a_t)
	\]
\end{defn}

The \textbf{\bluefont return} of a trajectory $(s_0,a_0,r_0,s_1,\ldots)$ at time $t$ is defined as
\begin{equation}
	G_t = \sum_{k} \gamma^k r_{t+k}
\end{equation}



\section{Decision rules, Policies}

The idea of a \textit{decision rule} is to choose which action (or actions, with a given preference) to take after observing part of an MDP's trajectory.









\end{document}
